\documentclass[a4paper,twoside,12pt,draft]{article}
%\documentclass[a4paper,twoside,12pt,draft]{article}

\usepackage[T1]{fontenc}
\usepackage[utf8]{inputenc}
\usepackage{amsthm,amsmath,amssymb,mathtools,xypic}
\usepackage{natbib} % Author–Year and Numerical Schemes citations

\newtheorem{theorem}{Theorem}
\newtheorem{proposition}{Proposition}
\theoremstyle{definition}
\newtheorem{definition}{Definition}
\newtheorem{problem}{Problem}
%\theoremstyle{example}
\newtheorem{example}{Example}
\theoremstyle{remark}
\newtheorem{remark}{Remark}

\DeclareMathOperator{\Ob}{Ob}
\DeclareMathOperator{\supp}{supp}

\newcommand{\dB}{\mathsf{dB}}

\newcommand{\Category}[1]{\mathsf{#1}}
\newcommand{\CCat}{\Category{C}}
\newcommand{\FCat}{\Category{F}}
\newcommand{\JCat}{\Category{J}}
\newcommand{\SetCat}{\Category{Set}}
\newcommand{\FinCat}{\Category{Fin}}

\newcommand{\NN}{\mathbb{N}}
\newcommand{\TT}{\mathtt{T}}
\newcommand{\MM}{\mathtt{M}}
\newcommand{\subst}{\mathtt{subst}}
\newcommand{\refe}{\mathtt{ref}}
\newcommand{\frees}{\mathtt{frees}}
\newcommand{\reindex}{\mathtt{reindex}}
\newcommand{\msubst}{\mathtt{msubst}}

\title{De Bruijn encoding as monadic restriction}
\author{Andr\'e Hirschowitz \and Marco Maggesi}
\date{August 2017}

\usepackage{color}
\usepackage[draft=false]{hyperref}
\hypersetup{
    bookmarks=true,         % show bookmarks bar?
    unicode=false,          % non-Latin characters in Acrobat’s bookmarks
    pdftoolbar=true,        % show Acrobat’s toolbar?
    pdfmenubar=true,        % show Acrobat’s menu?
    pdffitwindow=false,     % window fit to page when opened
    pdfstartview={FitH},    % fits the width of the page to the window
    pdftitle={My title},    % title
    pdfauthor={Author},     % author
    pdfsubject={Subject},   % subject of the document
    pdfcreator={Creator},   % creator of the document
    pdfproducer={Producer}, % producer of the document
    pdfkeywords={keyword1} {key2} {key3}, % list of keywords
    pdfnewwindow=true,      % links in new window
    colorlinks=true,       % false: boxed links; true: colored links
    linkcolor=blue,          % color of internal links
    citecolor=blue,        % color of links to bibliography
    filecolor=magenta,      % color of file links
    urlcolor=cyan           % color of external links
}

\begin{document}
\maketitle

\begin{abstract}
  We propose a high-level perspective on de Bruijn encoding
  which reposes on the theory of modules over a monad.
\end{abstract}

\tableofcontents{}

\section{Introduction}
\label{sec:intro}

The aim of this paper is to give a better understanding of de Bruijn
encoding of syntax with bindings by means of the more expressive and
abstract framework of (relative) monads and their modules.  In
particular:
\begin{itemize}
\item we introduce a new mathematical structure of \emph{de Bruijn
    monad} which axiomatizes parallel substitution and helps
  implementing and reasoning about the de Bruijn enconding of syntax
  with bindings;
\item we relate this new theory of de Bruijn monads with the already
  established theory of (relative) monad and modules;
\item we provide the construction of a term algebra and its associated
  catamorphism representing (in the sense of initial semantics) a
  given binding signature.
\end{itemize}

\subsection{The Problem of extending a relative monad}

Draft, draft, draft!

Let $\FCat$ be a subcategory of $\SetCat$, the category of sets.  We
will focus on the following two cases:
\begin{itemize}
\item $\FCat=\langle \NN \rangle$, the full subcategory of $\SetCat$
  having $\NN$ (the set of natural numbers) as its sole object.
\item $\FCat=\FinCat$, the category of finite sets.
\end{itemize}
We denote by $J\colon \FCat \hookrightarrow \SetCat$ the inclusion
functor and we consider the \emph{restriction} of $M$ to $\FCat$ given
by $M_\FCat = M \circ J$.  Then
$M_\FCat : \FCat \longrightarrow \SetCat$ has a structure of
\emph{relative monad} over $J$ as introduced by
\citet*{altenkirch_monads_2010}.

\begin{problem}[Extension of a relative monad]
  Given a relative monad $M$ over $J:\FCat \hookrightarrow \SetCat$,
  construct a monad $\hat M$ over $\SetCat$ such that its restriction
  $\hat M_\FCat$ is isomorphic to $M$.  When it exists, we call such
  $\hat M$, an \emph{extension} of $M$.
\end{problem}

\section{De Bruijn monads}
\label{sec:dbmonads}

This section gives a self-contained and direct presentation of the
theory of de Bruijn monads.  Later (TODO: give reference) we will
observe that de Bruijn monads can be introduced succinctly as a
special case of relative monads.

\subsection{Definition and examples}
\label{sec:def-dbmonad}

\begin{definition}[De Bruijn monads]
  \label{def:dbmonad}
  A \emph{de Bruijn monad} (or \emph{dB-monad}) is given by a type
  $\TT$ and two functions
  \begin{align*}
    \refe &\colon \NN \to \TT, \\
    \subst &\colon (\NN \to \TT) \to \TT \to \TT
  \end{align*}
  called \emph{reference} and \emph{substitution}, satisfying the
  following three conditions\footnote{For function application we
    follow the convention of implicit currying as in the HoTT Book,
    e.g., $f(x,y) = f(x)(y) = (f(x))(y)$.}
  \begin{align*}
    \subst(f,\refe(i)) &=  f(i)
    && \text{right unital law} \\
    \subst(\refe, x) &=  x
    && \text{left unital law} \\
    \subst(f, \subst(g, x)) &= \subst(\subst(f) \circ g, x)
    &&\text{associativity law}
  \end{align*}
  for all $x : \TT$, $i :\NN$ and $f,g\colon \NN \to \TT$.
\end{definition}

When no confusion can arise, we will denote simply $\TT$ the dB-monad
$\langle \TT, \refe, \subst \rangle$.   The type $\TT$ itself will be
referred as the \emph{carrier} of the dB-monad.

We now consider some basic examples of dB-monads.

\begin{example}[The initial dB-monad]
  \label{ex:initial-dbmonad}
  The type $\NN$ of natural numbers has a trivial structure of
  dB-monad with $\refe(i) \coloneqq i$ and
  $\subst(f,i) \coloneqq f(i)$.  This is the initial object in the
  category of dB-monads (Definition~\ref{def:dbmonad-morphism}).
\end{example}

\begin{example}[The dB-monad of lists]
  \label{ex:list-dbmonad}
  A less trivial example is the \emph{dB-monad of lists} defined as
  follows:
  \begin{align*}
    \TT &\coloneqq\mathtt{list}(\NN) \\
    \refe(i) &\coloneqq [i] \\
    \subst(f,[x_1, \dots, x_n] &\coloneqq f(x_1) \oplus \cdots \oplus f(x_n)
  \end{align*}
  where $[i]$ denotes the singleton lists and `$\oplus$' in the last
  identity denotes list concatenation.
\end{example}

As we already stressed in the introduction of this section, an
important class of examples is given by the following easy
construction.

\begin{definition}[Restriction of a total monad to a dB-monad]
  \label{def:dbmonad-restriction}
  Given a monad $M$ over $\SetCat$, consider the $M(\NN)$ endowed with
  $\refe = \eta_\NN$ and $\subst(f) \coloneqq \mathsf{bind}(f)$.  As
  immediate consequence of the axioms of monad, this gives us a
  dB-monad that we denote $M_\dB$ and call the \emph{restriction} of
  $M$.
\end{definition}

Indeed, our previous Example~\ref{ex:list-dbmonad} about the dB-monad of
lists is precisely the restriction of the monad of lists.

\begin{example}[dB-monad of $\lambda$-calculus]
  \label{ex:dbmonad-lambda}
  One key example that we will discuss thoroughly later (TODO:
  reference) is the dB-monad of $\lambda$-calculus, i.e., the
  restriction $\mathtt{\Lambda} \coloneqq \Lambda_\dB$ of the monad of
  $\lambda$-calculus as defined by (TODO: citation [HM], [Bird et
  al.]).
\end{example}

\subsection{Functoriality}
\label{sec:functoriality-dbmonad}

Given a dB-monad $\TT$ we can define the \emph{reindex} operator
\begin{equation*}
  \reindex\colon (\NN \to \NN) \to (\TT \to \TT)
\end{equation*}
given by
\begin{equation}
  \label{eq:reindex-as-subst}
  \reindex(f,x) = \subst(\refe \circ f, x).
\end{equation}
It is straightforward to verify, from the axioms of dB-monad, that the
following \emph{fusion} laws holds for $\reindex$:
\begin{align}
  \reindex(f,\refe(i)) &= \refe(f(i)) \\
  \reindex(\refe,x) &= x \\
  \reindex(f,\subst(g,x)) &= \subst(\reindex(f) \circ g, x) \\
  \subst(f,\reindex(g,x)) &= \subst(f \circ g,x) \\
  \reindex(f,\reindex(g,x)) &= \reindex(f\circ g, x)
\end{align}

\subsection{Morphisms}
\label{sec:morphisms-db-monads}

\begin{definition}[Morphisms of dB-monads and the $\dB$ category]
  \label{def:dbmonad-morphism}
  Given two dB-monads $\TT_1$, $\TT_2$ with substitutions $\subst_1$,
  $\subst_2$ and references $\refe_1$, $\refe_2$ a \emph{morphism} of
  dB-monads is a function $\phi\colon \TT_1 \to \TT_2$ that satisfies
  the identities
  \begin{align}
    \phi(\refe_1(i)) &= \refe_2(i)
        \label{eq:dbmonad-morphism-ref} \\
    \phi(\subst_1(f,x)) &= \subst_2(\phi \circ f, \phi(x))
        \label{eq:dbmonad-morphism-subst}
  \end{align}
  for all $i \in \NN$ and $f\colon \NN \to \TT_1$.

  It is immediate to verify that the previous definition builds
  dB-monads into a category $\dB$.
\end{definition}

%% TODO: which is precisely, the category of relative monads over the
%% functor $J$ of \eqref{eq:dB-functor}.

It follows at once that morphisms of dB-monads commute with reindexing
then, that is, for all $\phi\colon \TT_1 \to \TT_2$, $x\in \TT_1$ and
$f\colon \NN\to \NN$ we have
\begin{equation}
  \phi(\reindex_1(f,x)) = \reindex_2(f,\phi(x)) .
\end{equation}

\subsection{Free references}
\label{sec:free-references}

Let $\TT$ be a dB-monad.

\begin{definition}[Free references]
  We say that $i \in \NN$ is a \emph{free reference} of a term $x$ in
  $\TT$ if there exists a function $f\colon \NN \to \TT$ such that
  $f(j) = \refe(j)$ for all $j\neq i$ and $\subst(f,x) \neq x$.
\end{definition}

\begin{proposition}
  Let $x$ be a term of $\TT$ and $f,g:\NN\longrightarrow \TT$ be to
  two functions that agrees on $\frees{x}$.  Then
  \begin{equation*}
    \subst(f,x) = \subst(g,x)
  \end{equation*}
\end{proposition}

  The substitution operator determines the set
$\frees(x)$ of \emph{free references} in a term $x$ of a dB-monad
$\TT$.  The key property is
\begin{equation*}
  \subst(f,x) = x \Longleftrightarrow
  (\forall i.\, i \in \mathtt{frees}(x) \Longrightarrow f(i) = i).
\end{equation*}

One suitable definition of free reference is the following.  Given a
function $f: \NN \longrightarrow \TT$ we define its \emph{support} as
\begin{equation}
  \label{eq:support}
  \supp(f) = \big\{i\ \big|\ f(i)\neq i\big\}
\end{equation}

\begin{definition}[Free references]
  We say that the reference $i\in\NN$ \emph{occurs free} in $x\in\TT$
  if, for any funciton $f\colon\NN\longrightarrow \TT$ with support
  contained in $\{i\}$ we have
  \begin{equation*}
    \subst(f,x) = x.
  \end{equation*}
  We denote by $\mathtt{free}(x)$ the set of references that occurs
  free in $x$.
\end{definition}

For reindexing we have property analogous to that of substitution:
\begin{equation*}
  \reindex(f,x) = x \Longleftrightarrow
  (\forall i.\, i \in \mathtt{frees}(x) \Longrightarrow f(i) = \refe(i)).
\end{equation*}
More generally, we have the following result of extensionality.
\begin{proposition}
  Let $x$ be a term of a dB-monad $\TT$.  Then
  \begin{enumerate}
  \item if two functions $f,g\colon \NN\to \TT$ agrees on $\frees(x)$
    then $\subst(f,x) = \subst(g,x)$;
  \item if two functions $f,g\colon \NN\to \NN$ agrees on $\frees(x)$
    then $\reindex(f,x) = \reindex(g,x)$.
  \end{enumerate}
\end{proposition}

\section{Modules over dB-monads}
\label{sec:modules}

As for monad we have an associated notion of module, there is a
parallel notion of \emph{dB-module over dB-monads}.

\subsection{Definition of dB-module}
\label{sec:definition-module}

\begin{definition}[Modules of a dB-monad]
  A \emph{dB-module} (or simply \emph{module}) of a dB-monad $\TT$ is
  a type $\MM$ with an \emph{action}
  \begin{equation*}
    \msubst\colon (\NN\to \TT) \to \MM \to \MM
  \end{equation*}
  satisfying the laws
  \begin{align*}
    \msubst(\refe,x) &= x\\
    \msubst(f,\msubst(g,x)) &= \msubst(\subst f \circ g, x)
  \end{align*}
  for all $x\in \MM$ and $f,g\colon \NN \to \TT$.
\end{definition}

\subsection{Examples of modules}
\label{sec:examples-modules}

\begin{example}[The tautological dB-module]
  Every dB-monad is a module over itself in an obvious way.
\end{example}

\begin{example}[Modules associated to modules of a monad]
  Let $R$ be a monad over $\SetCat$.  Given a module $M$ over
  $R$, $M(\NN)$ has a natural structure of module over the dB-monad
  $R(\NN)$.
\end{example}

\begin{example}[Products of modules]
  Let $\TT$ be a dB-monad.  Arbitrary products of modules over $\TT$
  are, in an obvious way, modules over $\TT$ (see also
  Proposition~\ref{prop:modules-product}).
\end{example}

\subsection{Morphisms of dB-modules}
\label{sec:morphisms-db-modules}

In this section $\TT$ will be a fixed dB-monad.

\begin{definition}[Morphisms of dB-modules]
  Let $\MM_1$, $\MM_2$ two $\TT$-modules with actions $\msubst_1$ and
  $\msubst_2$.  A $\TT$-\emph{linear morphism} (or \emph{morphism of
    $\TT$-modules} is a map $\phi\colon \MM_1 \to \MM_2$ compatible
  with the module actions in the following sense:
  \begin{equation*}
    \phi(\msubst_1(f,x)) = \msubst_2(f,\phi(x)).
  \end{equation*}
\end{definition}

The family of modules over a fixed monad together with their morphisms
forms a category that we denote $\mathsf{Mod}(\TT)$.

\begin{proposition}
  \label{prop:modules-product}
  The category $\mathsf{Mod}(\TT)$ has arbitrary products.
\end{proposition}

\subsection{Derived module}
\label{sec:derived-module}

Let $\MM$ be a module over the dB-monad $\TT$.

\begin{definition}[Derived dB-module]
  The \emph{derived dB-module} $\MM'$ of $\MM$ is a dB-module over
  $\TT$ given by the following structure:
  \begin{itemize}
  \item the carrier of $\MM'$ is the same as that of $\MM$,
  \item the action $\msubst'$ of $\MM'$ is defined by
    \begin{equation*}
    \msubst'(f,x) \coloneqq \msubst(f',x)
  \end{equation*}
  where $f'$ is defined by the equations:
  \begin{align*}
    f'(0) &= \refe(0)\\
    f'(i+1) &= \reindex(\mathtt{suc},f(i))
  \end{align*}
  and $\mathsf{suc} \colon \NN \to \NN$ denotes the \emph{successor
    function} ($\mathtt{suc}(i) = i+1$).
  \end{itemize}
\end{definition}

The necessary verification to establish that $\MM'$ is indeed a module
are easy.  A refined statement is given by the following proposition.

\begin{proposition}(Derivation as endofunctor)
  \label{prop:derivation-endo}
  Derivation yields a Cartesian endofunctor of $\mathsf{Mod}(\TT)$.
\end{proposition}

\subsection{Pull-backs}
\label{sec:pull-backs}

\begin{definition}[Pull-back of a dB-module]
  Given a morphism of dB-module $\phi\colon \TT_1 \to \TT_2$ and a
  $\TT_2$ module $\MM$ with action $\msubst_2$ we can define a
  structure of $\TT_1$-module on $\MM$ given by the action
  \begin{equation*}
    \msubst_1(f,x) \coloneqq \msubst_2(\phi \circ f, x).
  \end{equation*}
  We will write $\phi^*(\MM)$ to denote the pull-back of $\MM$ along
  $\phi$.
\end{definition}

Given $\phi\colon \TT_1 \to \TT_2$ as in the definition every $\TT_2$
linear morphism $\psi\colon \MM_1 \to \MM_2$ is also a $\TT_2$-linear
morphism.  More precisely the following proposition holds.

\begin{proposition}
  \label{prop:pull-back-functor}
  The pull-back
  $\phi^* \colon \mathsf{Mod}(\TT_2) \to \mathsf{Mod}(\TT_1)$ is a
  cartesian functor.
\end{proposition}

\section{$\lambda$-calculus}
\label{sec:lambda-calculus}

\subsection{The dB-monad of $\lambda$-calculus}
\label{sec:db-monad-lambda}

We now describe the structure of dB-monad on the type $\mathtt{term}$
of $\lambda$-calculus.  The substitution is given by the
well-known \emph{parallel} substitution operator
\begin{equation*}
  \mathtt{subst}\colon (\NN \to \mathtt{term}) \to
  \mathtt{term} \to \mathtt{term}
\end{equation*}
and the unit is just the $\mathtt{Ref}$ construction.

The $\mathtt{App}$ and $\mathtt{Abs}$ constructions are linear
morphisms.  More precisely
\begin{equation*}
  \xymatrix{
    \mathtt{App} &\colon \mathtt{term} \times \mathtt{term} \to
                 \mathtt{term} \\
  \mathtt{Abs} &\colon \mathtt{term}' \to \mathtt{term}}
\end{equation*}

\begin{align*}
  \mathtt{App} &\colon \mathtt{term} \times \mathtt{term} \to
                 \mathtt{term} \\
\mathtt{Abs} &\colon \mathtt{term}' \to \mathtt{term}
\end{align*}
where $\mathtt{term}$ has to be considered as the tautological module
over itself, and $\mathtt{term}'$ denotes its derivative.

Finally, the single substitution can be easily derived from parallel substitution.
\begin{verbatim}
let subst1 u t  = subst (push u idenv) t;;
\end{verbatim}

\section{Relative monads}
\label{sec:relative-monads}

\begin{definition}[Relative monad]
  A \emph{relative monad} over a functor $J\colon \JCat\to \CCat$ is
  given by
  \begin{itemize}
  \item a mapping $M:\Ob \JCat \to \CCat$;
  \item for every object $X$ of $\JCat$, an arrow
    $\eta_X\colon J(X) \to M(X)$;
  \item for every pair of objects $X,Y$ of $\JCat$, and for every
    arrow $f\colon J(X) \to M(Y)$, a \emph{Kleisli extension}
    $\sigma(f) \colon M(X) \to M(Y)$
  \end{itemize}
  satisfying the identities
  \begin{align*}
    \sigma(f) \circ \eta &= f &&\text{right unital law} \\
    \sigma(\eta_X) &= \mathsf{id}_{M(X)} &&\text{left unital law} \\
    \sigma(f) \circ \sigma(g) &= \sigma(\sigma(f) \circ g) &&\text{associativity law}
  \end{align*}
  for all objects $X,Y,Z$ of $\JCat$ and forall arrows
  $f\colon J(Y)\to M(Z)$, $g\colon J(X)\to M(Y)$.
\end{definition}

We are going to show how a dB-monad $\TT$ has a natural structure of
relative monad.  Take
$\JCat \coloneqq\langle \NN \rangle$ (the full
subcategory of $\SetCat$ generated by $\NN$).  We call \emph{de
  Bruijn functor} (or \emph{dB-functor}) the inclusion
\begin{equation}
  \label{eq:dB-functor}
  \mathsf{dB}\colon \langle \NN \rangle \to \SetCat.
\end{equation}

Then, it can be observed that dB-monads correspond precisely to monads
relative to the de Bruijn functor.  More precisely, given a dB-monad
$\TT$, define
\begin{align*}
  M(\NN) &\coloneqq \TT, \\
  \eta_\NN &\coloneqq \refe, \\
  \sigma &\coloneqq \subst.
\end{align*}
Then $M,\eta,\sigma$ is a relative monad over $J$.

\section{Initial semantics with dB-monads}
\label{sec:init-semantics}

\subsection{Signatures and representations in dB-modules}
\label{sec:signatures-representations}

\begin{definition}[Arities and representations of arities]
  \hfill
  \begin{itemize}
  \item An \emph{arity} is a finite list of non-negative integer.
  \item Given a dB-monad $\TT$, a $\TT$-module $\MM$ and an arity
    $a=(a_1,\dots,a_n)$, a \emph{representation of the arity $a$ in
      $\MM$} is a $\TT$-linear morphism
    \begin{equation*}
      \phi:\MM^{(a_1)}\times\cdots\times\MM^{(a_n)} \longrightarrow \MM
    \end{equation*}
  \item A \emph{signature} is a family of arities.
  \item A \emph{representation of a signature} $\Sigma$ is given by a
    representation for each of its arities.
  \end{itemize}
\end{definition}

In the sequel, we will abbreviate
$\MM^{(a_1)}\times\cdots\times\MM^{(a_n)}$ with $M^{a_1,\dots,a_n}$ or
by $\MM^a$.

In particular, we can consider representation of a signature in a
dB-monad, by taking the associated tautological module.  We form a
category of representation in a dB-monad.

\begin{definition}[The category of representation in a dB-monad]
  Let $\phi \colon \TT_1 \to \TT_2$ be a morphism between two
  dB-monads, each of which endowed by a representation of a signature
  $\Sigma$.  We say that $\phi$ is a \emph{morphism of
    $\Sigma$-representations} if, for each arity $a$ of $\Sigma$, the
  induced diagram diagram
  \begin{equation*}
    \xymatrix{
      \TT_1^a\ar[d]\ar[r] & \TT_1\ar[d]^\phi \\
      \TT_2^a \ar[r] & \TT_2}
  \end{equation*}
  where the orizontal arrows are the representations of $a$.
\end{definition}

\bibliographystyle{plainnat}
%\bibliographystyle{abbrvnat}
\bibliography{dbmonad}
\end{document}

%  LocalWords: unital associativity monad monoid functor monads arity
%  LocalWords: arities morphism morphisms monadic
%  LocalWords: extensionality
